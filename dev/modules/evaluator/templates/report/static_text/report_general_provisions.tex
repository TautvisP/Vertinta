\section{Bendrosios Nuostatos}


\textbf{Turto vertinimo subjektai}

\begin{enumerate}
    \item Turto vertinimo subjektai yra užsakovai ir turtą vertinančios įmonės, taip pat Vyriausybės arba miestų (rajonų) valdybų (merų) įgaliotos institucijos.
    \item Užsakovais gali būti fiziniai, juridiniai asmenys ar įmonės, neturinčios juridinio asmens teisių, norintys žinoti turto vertę.
    \item Turto vertintojai gali verstis turto vertinimo veikla būdami juridinio asmens teisių neturinčios įmonės savininkais ar tikraisiais nariais arba dirbdami darbo sutarties pagrindais įmonėse, Vyriausybės arba miestų (rajonų) valdybų (merų) įgaliotose institucijose.
\end{enumerate}



\textbf{Turto vertinimo objektai}
    Turto vertinimo objektas yra bet koks Lietuvos Respublikos teritorijoje esantis materialus ir nematerialus turtas, verslas, taip pat teisė į turtą (arba turto dalį), jeigu ši teisė gali būti perleidžiama kitiems asmenims.



\textbf{Turto vertinimo atvejai}

\begin{enumerate}
    \item Turtas gali būti vertinamas, kai:
    \begin{enumerate}
        \item keičiasi jo savininkas, t. y. turtas parduodamas, perduodamas kaip nepiniginis (turtinis) įnašas, mainomas, dovanojamas, paveldimas;
        \item jis apdraudžiamas;
        \item jis apmokestinamas, deklaruojamas;
        \item jis jungiamas su kitu turtu, padalijamas ar atidalijamas iš bendro turto;
        \item jis įkeičiamas;
        \item jis įrašomas į finansinės apskaitos dokumentus;
        \item jis išnuomojamas arba perduodamas kitiems asmenims pasaugos, panaudos pagrindais ar patikėjimo teise;
        \item jis paimamas įstatymų nustatyta tvarka visuomenės poreikiams;
        \item jis pripažįstamas bešeimininkiu;
        \item įmonėms taikomos bankroto procedūros;
        \item vykdomi teismų sprendimai, nutartys ir nutarimai civilinėse bylose, teismų nuosprendžiai ir nutartys baudžiamosiose bylose dėl turtinių išieškojimų, hipotekos teisėjų nutartys įkeistą turtą parduoti iš varžytynių;
        \item to pageidauja turto savininkas arba turto vertinimo užsakovas.
    \end{enumerate}
    \item Įstatymų nustatytais atvejais turto vertinimas yra privalomas ir atliekamas šio įstatymo 7 straipsnyje nurodytais metodais. Išimtiniais atvejais Vyriausybė gali nustatyti ir kitokius vertinimo metodus.
\end{enumerate}



\textbf{Turto vertės nustatymo principai}

\begin{enumerate}
    \item Turto vertė nustatoma vadovaujantis rinkos ekonomikos logika bei kriterijais, rinkos ir ekonominių sąlygų tyrimų bei stebėjimų rezultatais.
    \item Turto vertė nustatoma laikantis apdairumo ir atsargumo siekiant nepažeisti turto savininko teisių, taip pat turto pakeitimo kitu turtu bei alternatyvaus turto panaudojimo principų.
    \item Turto vertės nustatymo pagrindas yra galimos pajamos bei pelnas turtą naudojant ar juo disponuojant arba asmeninių poreikių tenkinimas.
    \item Turto vertė gali būti nustatoma pagal:
    \begin{enumerate}
        \item pirkimo-pardavimo sandorius;
        \item turto sukūrimo (atkūrimo), įsigijimo kaštus;
        \item pajamas, gaunamas naudojant turtą.
    \end{enumerate}
\end{enumerate}



\textbf{Turto vertinimo metodai}

Taikomi šie turto vertinimo metodai arba jų deriniai:

\begin{enumerate}
    \item lyginamosios vertės (pardavimo kainos analogų) metodas (toliau – lyginamosios vertės metodas), kurio esmė yra palyginimas, t. y. turto rinkos vertė nustatoma palyginus analogiškų objektų faktinių sandorių kainas, kartu atsižvelgiant į nedidelius vertinamo turto bei jo analogo skirtumus;
    \item atkuriamosios vertės (kaštų) metodas (toliau – atkuriamosios vertės metodas), kurio pagrindas yra skaičiavimai, kiek kainuotų atkurti esamos fizinės būklės ir esamų eksploatacinių bei naudingumo savybių objektus pagal vertinimo metu taikomas darbų technologijas bei kainas;
    \item naudojimo pajamų vertės (pajamų kapitalizavimo arba pinigų srautų diskonto) metodas (toliau – naudojimo pajamų vertės metodas), kai turtas vertinamas ne kaip įvairaus turto suma, bet kaip verslo objektas, duodantis pelną. Šis metodas taikomas tada, kai tikimasi, kad vertinamo turto naudojimo vertė objektyviausiai parodys turto vertę rinkoje;
    \item ypatingosios vertės metodas – unikalioms meno ir istorijos vertybėms, juvelyriniams ir antikvariniams dirbiniams, įvairioms kolekcijoms įvertinti (vertinama pagal specialias šio turto vertinimo technologijas);
    \item kiti Lietuvos Respublikos Vyriausybės aprobuoti ir tarptautinėje praktikoje pripažinti metodai.
\end{enumerate}

Parengta remiantis Lietuvos Respublikos turto ir verslo vertinimo pagrindų įstatymu Nr. VIII-1202 įsigaliojusiu nuo 1999 m. gegužės 25 d.

\pagebreak