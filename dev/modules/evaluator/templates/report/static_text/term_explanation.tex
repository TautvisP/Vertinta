\section{Terminų paaiškinimai}

\begin{enumerate}
    \item \textbf{Turtas} – materialios, nematerialios ir finansinės vertybės.

    \item \textbf{Nekilnojamasis turtas} – žemė ir su ja susiję objektai, kurių buvimo vietos negalima pakeisti, nekeičiant jų naudojimo paskirties arba nemažinant vertės bei ekonominės paskirties, arba turtas, kuris tokiu pripažįstamas įstatymuose.

    \item \textbf{Kilnojamasis turtas} – turtas, kurį galima perkelti iš vienos vietos į kitą nepakeitus jo turinio, iš esmės nesumažinus jo vertės ar be didelės žalos jo paskirčiai, jeigu įstatymai nenustato ko kita.

    \item \textbf{Verslas} – juridinio ar fizinio asmens, įmonės, neturinčios juridinio asmens teisių, veikla, kuriai naudojami ekonominiai ištekliai, kuria siekiama gauti pajamų bei pelno ir už kurią šis asmuo atsako savo turtu.

    \item \textbf{Turto ar verslo (toliau – turto) vertintojas} – fizinis asmuo, turintis Lietuvos Respublikos audito, apskaitos ir turto vertinimo instituto išduotą turto vertintojo kvalifikacijos pažymėjimą ir besiverčiantis turto vertinimo veikla.

    \item \textbf{Turtą ar verslą (toliau – turtą) vertinanti įmonė} – įmonė, įregistruota Įmonių rejestro įstatymo nustatyta tvarka ir turinti Lietuvos Respublikos audito, apskaitos ir turto vertinimo instituto išduotą kvalifikacijos atestatą, suteikiantį teisę verstis turto vertinimo veikla.

    \item \textbf{Užsakovas} – juridinis, fizinis asmuo arba įmonė, neturinti juridinio asmens teisių, sudarę su turtą vertinančia įmone turto vertinimo sutartį.

    \item \textbf{Turto ar verslo (toliau – turto) vertinimas} – nešališkas turto vertės nustatymas taikant šio įstatymo 7 straipsnyje nurodytus turto vertės nustatymo metodus ir apimantis vertintojo nuomonę apie turto būklę, jo tinkamumą naudoti bei tikėtiną piniginę vertę rinkoje.

    \item \textbf{Individualus turto ar verslo (toliau – turto) vertinimas} – toks turto vertinimo būdas, kai konkretaus turto vertė nustatoma atsižvelgiant į visas individualias to turto savybes.

    \item \textbf{Masinis turto ar verslo (toliau – turto) vertinimas} – toks turto vertinimo būdas, kai konkretaus turto vertė nėra nustatoma, o surinktos informacijos apie vertinamąjį turtą analizės būdu nustatomos verčių ribos, apimančios vertinamojo turto vertę. Duomenys renkami, analizuojami ir apskaičiavimai atliekami sisteminimo pagrindu. Šiuo vertinimo būdu yra vertinami turto objektai, kurie turi daug panašumų.

    \item \textbf{Vertė} – prekių (paslaugų) ar kito turto, ar verslo naudingumo tam tikru metu matas, nustatytas pagal atitinkamą vertinimo metodą.

    \item \textbf{Kaina} – pinigų suma, kuri yra paprašyta, pasiūlyta ar sumokėta už prekes (paslaugas) ar kitą turtą. Kaina už konkrečias prekes (paslaugas) ar kitą turtą yra reliatyvus vertės patvirtinimas, padarytas konkrečių pardavėjų (paslaugų teikėjų) ir pirkėjų (paslaugų gavėjų) tam tikromis aplinkybėmis.

    \item \textbf{Rinkos vertė} – apskaičiuota pinigų suma, už kurią galėtų būti parduotas turtas vertinimo dieną, sudarius tiesioginį komercinį sandorį tarp norinčių turtą parduoti ir norinčių turtą pirkti asmenų po šio turto tinkamo pateikimo į rinką, jeigu abi sandorio šalys veiktų dalykiškai, be prievartos ir nesąlygojamos kitų sandorių bei interesų.

    \item \textbf{Atkuriamoji vertė} – apskaičiuota pinigų suma (kaštai), kurios reikėtų tokių pat fizinių ir eksploatacinių savybių objektui sukurti, pagaminti arba pastatyti (įrengti).

    \item \textbf{Naudojimo vertė} – apskaičiuota pinigų suma, išreiškianti turto ekonominį naudingumą (turto teikiamą ekonominę naudą).

    \item \textbf{Specialios paskirties turtas} – turtas, kuris retai parduodamas arba visai neparduodamas, išskyrus atvejus, kai jis yra viso verslo sudedamoji dalis. Atsižvelgiant į specialų panaudojimą, toks objektas laikomas ribotos paklausos arba ne rinkai skirtu turtu.
\end{enumerate}

Parengta remiantis Lietuvos Respublikos turto ir verslo vertinimo pagrindų įstatymu Nr. VIII-1202 įsigaliojusiu nuo 1999 m. gegužės 25 d.

\pagebreak