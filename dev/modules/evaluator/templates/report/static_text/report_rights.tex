\section{Turto vertintojai ir teisės aktai}


\textbf{Turto vertintojų teisės}
Turto vertintojas turi šias teises:
\begin{enumerate}
    \item nepažeisdamas įstatymų, gauti iš užsakovo ir kitų informacijos šaltinių informaciją bei duomenis, reikalingus turto vertinimui atlikti, taip pat naudotis valstybės ir valdžios institucijų bei įmonių vieša informacija, reikalinga turto vertinimui atlikti;
    \item patikrinti vertinamą turtą, jo fizinę ir ekonominę būklę, teises į vertinamą turtą, nuosavybės teisių apribojimus ir kitas turto charakteristikas, reikšmingas turto vertei nustatyti;
    \item suderinęs su užsakovu, pasitelkti reikalingus specialistus bei ekspertus vertinimui atlikti.
\end{enumerate}



\textbf{Turtą vertinančių įmonių ir vertintojų pareigos}
Savo veikloje turtą vertinanti įmonė ir vertintojas privalo:   
\begin{enumerate}
    \item vadovautis įstatymais ir kitais teisės aktais, Turto vertintojo profesinės etikos kodeksu;
    \item sąžiningai ir rūpestingai atlikti pareigas, profesionaliai atlikti vertinimą;
    \item laikyti paslaptyje komercinę informaciją, kurią sužinojo atlikdamas turto vertinimą, ir nenaudoti jos asmeninei ar kitų asmenų naudai;
    \item šio straipsnio 3 punkte nurodytą informaciją atskleisti tik įstatymų nustatytais atvejais ir tvarka;
    \item visais atvejais pateikti užsakovui įmonės, atliekančios turto vertinimą, kvalifikacijos atestatą, įmonės civilinės atsakomybės draudimo polisą, turto vertintojo kvalifikacijos pažymėjimą.
\end{enumerate}



\textbf{Žalos atlyginimas}
Turtą vertinanti įmonė atsako už atlikto vertinimo teisingumą, jo atlikimo terminus teisės aktų ir sutartyje tarp įmonės bei užsakovo nustatyta tvarka, taip pat atlygina užsakovui žalą, atsiradusią dėl netinkamai atliktų įsipareigojimų.



\textbf{Turtą vertinančios įmonės atgręžtinio reikalavimo teisė}
Atlyginusi žalą, turtą vertinanti įmonė turi atgręžtinio reikalavimo teisę į kaltus darbuotojus, kurie privalo atlyginti įmonei žalą darbo įstatymų nustatytu dydžiu ir tvarka.

 

\textbf{Turtą vertinančios įmonės civilinės atsakomybės draudimas}
Turtą vertinanti įmonė, išskyrus Vyriausybės arba miestų (rajonų) valdybų (merų) įgaliotą instituciją, privalo apsidrausti civilinės atsakomybės draudimu. Vyriausybė nustato minimalų civilinės atsakomybės draudimo sumos dydį, atsižvelgdama į įmonių pajamas, gautas už turto vertinimą, ir turto vertintojų kvalifikaciją.



\textbf{Turto vertintojo teisių vertinti turtą apribojimas}
\begin{enumerate}
    \item Turto vertintojui draudžiama vertinti turtą, jeigu:
    \begin{enumerate}
        \item turto vertintojas yra vertinamo turto savininkas (bendrasavininkis);
        \item turto vertintojas yra susijęs šeimos ar giminystės ryšiais – tėvai (įtėviai), vaikai (įvaikiai), senoliai, broliai, seserys, vaikaičiai, provaikaičiai, proseneliai ir prosenelės, brolio ir sesers vaikai (sūnėnai ir dukterėčios), tėvo ar motinos broliai ir seserys (dėdės ir tetos), tėvo ar motinos brolių ir seserų vaikai (pusbroliai ir pusseserės) – su vertinamo turto savininku (bendrasavininkiu) ar patikėjimo teise valdančiais turtą įmonių, įstaigų ir organizacijų vadovais;
        \item atsiranda kitos sąlygos, galinčios turėti įtakos turto vertintojo nepriklausomumui, nešališkumui ir objektyvumui.
    \end{enumerate}
    \item Turto vertintojas gali vertinti tik tą turto sritį, kuri yra nurodyta turto vertintojo kvalifikacijos pažymėjime.
\end{enumerate}



\textbf{Dokumentų, patvirtinančių teisę vertinti turtą, išdavimas, jų galiojimo sustabdymas ir (ar) panaikinimas}
Dokumentai, patvirtinantys teisę verstis turto vertinimo veikla, yra:
\begin{enumerate}
    \item turto vertintojo kvalifikacijos pažymėjimas, kuris išduodamas, jo galiojimas sustabdomas arba panaikinamas Vyriausybės arba jos įgaliotos institucijos nustatyta tvarka;
    \item įmonės kvalifikacijos atestatas, leidžiantis verstis turto vertinimo veikla, kuris suteikiamas, jo galiojimas sustabdomas ir panaikinamas Vyriausybės arba jos įgaliotos institucijos nustatyta tvarka.
\end{enumerate}

Parengta remiantis Lietuvos Respublikos turto ir verslo vertinimo pagrindų įstatymu Nr. VIII-1202 įsigaliojusiu nuo 1999 m. gegužės 25 d.

\pagebreak