\pgfsetlayers{background,main,foreground}

\section{Vertinimo Reglamentavimas}

\textbf{Vertinimo reikalavimai}

\begin{enumerate}
    \item Vertinimas atliekamas vadovaujantis šiuo įstatymu, kitais teisės aktais, kuriuose nustatomas reikalavimas atlikti vertinimą, ir Tarptautiniais vertinimo standartais arba Europos vertinimo standartais. Tarptautiniais vertinimo standartais arba Europos vertinimo standartais vadovaujamasi tiek, kiek jie neprieštarauja šio įstatymo ir kitų teisės aktų, kuriuose nustatomas reikalavimas atlikti vertinimą, nuostatoms.
    \item Vertinimą atlieka vertintojas, turintis galiojančią civilinės atsakomybės draudimo, kurio minimali civilinės atsakomybės draudimo suma yra 60 000 eurų vienam draudžiamajam įvykiui ir 150 000 eurų visiems draudžiamiesiems įvykiams per metus, sutartį.
\end{enumerate}

\textbf{Vertintojo nepriklausomumas}

\begin{enumerate}
    \item Vertintojas turi būti nepriklausomas. Vertintojas nelaikomas nepriklausomu, jeigu jis yra:
    \begin{enumerate}
        \item vertinamo objekto savininkas arba bendraturtis;
        \item vertinimo užsakovo, naudotojo, kuris yra žinomas vertinimo atlikimo metu, arba vertinamo objekto savininko arba bendraturčio sutuoktinis, artimos giminystės, svainystės ar partnerystės ryšiais susijęs asmuo;
        \item vertinimo užsakovo, naudotojo arba juridinio asmens, kuris nuosavybės arba patikėjimo teise valdo vertinamą objektą, dalyvis, organų narys.
    \end{enumerate}
    \item Vertintojo juridinio asmens dalyviams, organų nariams taikomi šio straipsnio 1 dalyje nustatyti nepriklausomumo kriterijai.
\end{enumerate}

\textbf{Vertintojo kvalifikacija}

\begin{enumerate}
    \item Vertintojo kvalifikacija pagrindžiama Vertintojų rūmams pateikiant:
    \begin{enumerate}
        \item pažymėjimą, kuriuo patvirtinama, kad išlaikytas vertintojo kvalifikacijos egzaminas;
        \item pažymėjimą, kuriuo patvirtinama, kad per kalendorinius metus išklausytas ne trumpesnis negu 20 valandų kvalifikacijos tobulinimo kursas vertinimo temomis.
    \end{enumerate}
        \item Vertintojo kvalifikacijos egzaminą ir vertintojo kvalifikacijos tobulinimą organizuoja ir kvalifikacijos tobulinimo priežiūrą atlieka Vertintojų rūmai.
        \item Vertintojų rūmai, siekdami informuoti visuomenę apie asmenis, turinčius vertintojo kvalifikaciją, tvarko ir savo interneto svetainėje skelbia vertintojo kvalifikaciją turinčių asmenų sąrašą, nurodydami vardus ir pavardes.
        \item Vertintojo kvalifikacijos egzaminas ir vertintojų kvalifikacijos tobulinimo priežiūra organizuojami, vertintojo kvalifikaciją turinčių asmenų sąrašas skelbiamas ir tvarkomas Vertintojų rūmų nustatyta tvarka.
        \item Vertintojų rūmai, organizuodami vertintojo kvalifikacijos suteikimą ir vertintojo kvalifikaciją turinčių asmenų sąrašo skelbimą ir tvarkymą, turi užtikrinti savo veiklos tęstinumą ir kartą per metus pateikti Tarnybai informaciją apie savo funkcijų atlikimą. Tarnybos prašymu ši informacija turi būti teikiama ir dažniau.
\end{enumerate}

\textbf{Vertintojo teisė gauti duomenis ir informaciją}

Vertintojas turi teisę gauti iš vertinimo užsakovo, valstybės ir savivaldybės institucijų ir įstaigų, įmonių ir viešųjų įstaigų bei valstybės ir savivaldybės valdomų subjektų, nurodytų Lietuvos Respublikos teisės gauti informaciją ir duomenų pakartotinio naudojimo įstatymo 2 straipsnio 1 dalies 1 ir 2 punktuose, taip pat valstybės informacinių sistemų valdytojų, valstybės informacinių sistemų tvarkytojų, registrų valdytojų, registrų tvarkytojų informaciją ir duomenis (įskaitant asmens duomenis), kurie būtini vertinimui atlikti.


\textbf{Privalomojo turto arba verslo vertinimo ataskaitos reikalavimai}
\begin{enumerate}
    \item Vertintojas, atlikęs vertinimą, parengia privalomojo turto arba verslo vertinimo ataskaitą (toliau – vertinimo ataskaita) pagal Tarptautinių vertinimo standartų arba Europos vertinimo standartų reikalavimus.
    \item Vertinimo ataskaitoje nurodoma, kokiais teisės aktais ir Tarptautiniais vertinimo standartais ar Europos vertinimo standartais buvo vadovaujamasi atliekant vertinimą.
    \item Vertinimo ataskaitą pasirašo vertinimą atlikęs vertintojas fizinis asmuo ir, jeigu su vertinimo užsakovu sutartį sudarė vertintojas juridinis asmuo, šio juridinio asmens vadovas arba jo įgaliotas atstovas.
\end{enumerate}

Parengta remiantis Lietuvos Respubliko privalomojo turto ir verslo vertinimo įstatymu Nr. XIV-2575 paruoštu 2024 m. balandžio 25 d.

\pagebreak